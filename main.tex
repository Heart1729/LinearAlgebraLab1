\documentclass[12pt]{article}
\usepackage{graphicx}

% Русский язык и нормальные шрифты
\usepackage[utf8]{inputenc}
\usepackage[russian]{babel}
\usepackage[T2A]{fontenc}

% Математика
\usepackage{amsmath, amssymb, amsthm}

% Поля
\usepackage{geometry}
\geometry{a4paper, margin=2cm}

% Нумерация
\numberwithin{equation}{section}

% Окружения
\theoremstyle{definition}
\newtheorem{definition}{Определение}
\newtheorem{theorem}{Теорема}
\newtheorem{lemma}{Лемма}
\newtheorem{proposition}{Утверждение}

% Команды для удобства
\newcommand{\F}{\mathbb{F}}
\newcommand{\Z}{\mathbb{Z}}

\begin{document}

$N = 4$, $p = 11$, $m = 2$

$a_i = (i+4)$ mod $4$

$b_j = (j+4)$ mod $7$

$c_k = (k+4)$ mod $5$

$d_l = (l+4)$ mod $9$

$r_m = (m+4)$ mod $11$

$s_t = (t+4)$ mod $11$


\section*{Задача 1}

\textbf{Решение.}

$$
x^9 + a_8x^8 + ...+a_1x_1+a_0 = 0
$$

Подставляем значение коэфициентов.

$$
x^9 + 3x^7 + 2x^6 + x^5 + 3x^3 + 2x^2 + x = 0
$$

Корни принадлежат $F_4[x]$. Он содержит $4$ элемента: $0$, $1$, $a$, $a + 1$, такие, что $a^1 = a$, $a^2 = a + 1$, $a^3 = 1$. Поле $F_4[x]$ имеет характеристику 2, значит $1 + 1 = 0.$

Применим вышеописанное для решения задачи.

$$
x^9 + 3x^7 + 2x^6 + x^5 + 3x^3 + 2x^2 + x = 0
$$
Применим 1 + 1 = 0.
$$
x^9 + x^7 + x^5 + x^3 + x = 0
$$
Подставим x = 0. Он будет корнем.
$$
0 = 0
$$
Подставим x = 1. Он не будет корнем.
$$
1 + 1 + 1 + 1 + 1 = 1 \neq 0
$$
Подставим x = a. Он не будет корнем
$$
a^9 + a^7 + a^5 + a^3 + a = (a^3)^3 + (a^3)^2 \cdot a + a^3 \cdot a^2 + a^3 + a = 1 + a + a^2 + 1 + a = a + 1 \neq 0
$$

Подставим $x = a + 1,$ зная, что $(a + 1)^2 = a$, $(a + 1)^3 = 1$. Он не будет корнем.
$$
(a + 1)^9 + (a + 1)^7 + (a + 1)^5 + (a + 1)^3 + (a + 1)
$$
$$
((a + 1)^3)^3 + ((a + 1)^3)^2 \cdot (a + 1) + (a + 1)^3 \cdot (a + 1)^2 + (a + 1)^3 + (a + 1) = 1 + (a + 1) + (a + 1)^2 + 1 + (a + 1)
$$
$$
1 + a + 1 + a + 1 + a + 1 = a \neq 0
$$

Значит в $F_4$ корень один и он равен нулю.


Найти корни многочлена в $F_7[x]$:
$$
3x^6 + 2x^5 + x^4 + 6x^2 + 5x + 4 = 0.
$$

Поскольку корни должны принадлежать $F_7$, переберём все элементы поля:
$$
F_7 = \{0, 1, 2, 3, 4, 5, 6\}.
$$

Подставим $x = 0$:
$$
4 \neq 0.
$$
$x = 0$ не корень.

Подставим $x = 1$:
$$
3\cdot1^6 + 2\cdot1^5 + 1^4 + 6\cdot1^2 + 5\cdot1 + 4 =
3 + 2 + 1 + 6 + 5 + 4 = 21.
$$
$$
21 \equiv 0 \pmod{7}.
$$
Значит $x = 1$ — корень.

Подставим $x = 2$:
$$
3\cdot1 + 2\cdot4 + 2 + 6\cdot4 + 5\cdot2 + 4 =
3 + 8 + 2 + 24 + 10 + 4 = 51 \text{ не кратно 7}.
$$
$x = 2$ не корень.

Подставим $x = 3$:
$$
3\cdot1 + 2\cdot5 + 4 + 6\cdot2 + 5\cdot3 + 4 =
3 + 10 + 4 + 12 + 15 + 4 = 48 \text{ не кратно 7}.
$$
$x = 3$ не корень.

Подставим $x = 4$:
$$
3\cdot1 + 2\cdot2 + 4 + 6\cdot2 + 5\cdot4 + 4 =
3 + 4 + 4 + 12 + 20 + 4 = 47 \text{ не кратно 7}.
$$
$x = 4$ не корень.

Подставим $x = 5$:
$$
3\cdot1 + 2\cdot3 + 2 + 6\cdot4 + 5\cdot5 + 4 =
3 + 6 + 2 + 24 + 25 + 4 = 64 \text{ не кратно 7}.
$$
$x = 5$ не корень.

Подставим $x = 6$:
Тогда:
$$
3\cdot1 + 2\cdot(-1) + 1 + 6\cdot1 + 5\cdot(-1) + 4 =
3 - 2 + 1 + 6 - 5 + 4 = 7 \text{ кратно 7}.
$$
Значит $x = 6$ — корень.

Мы получили корни $1$ и $6$.

\section*{Задача 2}

\textbf{Решение.}
Рассмотрим многочлен над $F_5$:
$$
f(x)=x^5+3x^4+2x^3+x^2+4.
$$

Проверим наличие линейных множителей, то есть корней в $F_5=\{0,1,2,3,4\}$.

Подставим $x = 0$:
$$
f(0)=4 \neq 0.
$$

Подставим $x = 1$:
$$
1 + 3 + 2 + 1 + 4 = 11 \equiv 1 \pmod{5} \neq 0.
$$

Подставим $x = 2$.
Заметим:
$$
2^2=4,\;2^3=3,\;2^4=1,\;2^5=2.
$$
Тогда:
$$
f(2)=2 + 3\cdot1 + 2\cdot3 + 4 + 4 = 2 + 3 + 6 + 4 + 4 = 19 \equiv 4 \neq 0.
$$

Подставим $x = 3$.
Заметим:
$$
3^2=4,\;3^3=2,\;3^4=1,\;3^5=3.
$$
Тогда:
$$
f(3)=3 + 3\cdot1 + 2\cdot2 + 4 + 4 = 3 + 3 + 4 + 4 + 4 = 18 \equiv 3 \neq 0.
$$

Подставим $x = 4 \equiv -1$.
Заметим:
$$
(-1)^2=1,\;(-1)^3=4,\;(-1)^4=1,\;(-1)^5=4.
$$
Тогда:
$$
f(4)=4 + 3\cdot1 + 2\cdot4 + 1 + 4 = 4 + 3 + 8 + 1 + 4 = 20 \equiv 0 \pmod{5}.
$$

Значит $x=4$ является корнем, следовательно $(x-4)$ является множителем. В $F_5$ число $-4 \equiv 1$, поэтому:
$$
x-4 \equiv x+1.
$$

Разделим $f(x)$ на $(x+1)$:
$$
f(x) = (x+1)(x^4 + 2x^3 + x + 4).
$$

Исследуем многочлен
$$
g(x)=x^4+2x^3+x+4
$$
на приводимость.

Проверим, есть ли корни:
$$
g(0)=4\neq 0,\quad
g(1)=1+2+1+4=8\equiv 3\neq 0,
$$
$$
g(2)=1+6+2+4=13\equiv 3\neq 0,\quad
g(3)=1+4+3+4=12\equiv 2\neq 0,
$$
$$
g(4)=1+3+4+4=12\equiv 2\neq 0.
$$

Корней нет, значит линейных множителей нет.

Следовательно, $g(x)$ неприводим над $F_5$.

$$
f(x) = (x+1)(x^4 + 2x^3 + x + 4),\quad x^4+2x^3+x+4 \text{ неприводим в } F_5[x].
$$

\section*{Задача 3}

\textbf{Решение.}

Рассмотрим многочлены над $F_{11}$:
$$
f(x)=10x^6+9x^5+8x^4+7x^3+6x^2+5x+4,
\qquad
g(x)=7x^3+6x^2+5x+4.
$$

Применим алгоритм Евклида:

$$
f(x) = (x^3+2x^2+3x+1)\cdot g(x) + 1.
$$

Поскольку остаток равен 1, имеем:
$$
\gcd(f(x),g(x)) = 1.
$$

Отсюда следует, что $f(x)$ и $g(x)$ взаимно просты в $F_{11}[x]$.

Теперь найдём линейное представление $\gcd(f,g)$:
$$
1 = f(x) - (x^3+2x^2+3x+1)\cdot g(x).
$$

Таким образом,
$$
u(x)=1,\qquad v(x)=-(x^3+2x^2+3x+1)\equiv (10x^3+9x^2+8x+10)\pmod{11}.
$$

$$
\gcd(f,g)=1, \quad 1 = f(x) + (10x^3+9x^2+8x+10)\,g(x) = f(x)u(x) + v(x)g(x).
$$

\section*{Задача 4}

\textbf{Решение.}
$$
\text{Работаем над полем } F_{13}.
$$

$$
f(x) = 6x^2 + 5x + 4 \in F_{13}[x].
$$

$$
g(x) = x^8 + x^4 + x^3 + 6x + 2 \in F_{13}[x].
$$

Требуется найти обратный элемент \(f^{-1} \pmod{g(x)}\), то есть многочлен \(h(x)\), удовлетворяющий
$$
f(x)h(x) \equiv 1 \pmod{g(x)}.
$$


Расширенный алгоритм Евклида.

Зададим начальные значения:
$$
r_0 = g(x), \quad r_1 = f(x), \qquad
u_0 = 1,\ v_0 = 0,\qquad
u_1 = 0,\ v_1 = 1.
$$

Делим:
$$
r_0 = q_1 r_1 + r_2,
$$
где
$$
q_1(x) = -2x^6 + 6x^5 + 5x^4 - 6x^3 + 4x^2 + 3x - 3,
\qquad
r_2(x) = -4x + 1.
$$

Обновляем коэффициенты:
$$
u_2 = u_0 - q_1 u_1, \qquad v_2 = v_0 - q_1 v_1.
$$

Следующее деление:
$$
r_1 = q_2 r_2 + r_3,
\qquad
q_2(x)=5x, \quad r_3(x)=4.
$$
$$
u_3 = u_1 - q_2 u_2, \qquad v_3 = v_1 - q_2 v_2.
$$

Последнее деление:
$$
r_2 = q_3 r_3 + 0,
\qquad q_3(x)= -x - 3.
$$

Последний ненулевой остаток:
$$
\gcd(f,g)=4.
$$

Так как $(4 \in F_{13})$, домножаем результат на $(4^{-1} \equiv 10 \pmod{13}).$ 
Отсюда получаем искомый обратный элемент:
$$
h(x) = 4x^7 + x^6 + 3x^5 - x^4 + 5x^3 - 6x^2 + 6x - 3.
$$

\end{document}
