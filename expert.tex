\documentclass[12pt]{article}
\usepackage{graphicx}

% Русский язык и нормальные шрифты
\usepackage[utf8]{inputenc}
\usepackage[russian]{babel}
\usepackage[T2A]{fontenc}

% Математика
\usepackage{amsmath, amssymb, amsthm}

% Поля
\usepackage{geometry}
\geometry{a4paper, margin=2cm}

% Нумерация
\numberwithin{equation}{section}

% Окружения
\theoremstyle{definition}
\newtheorem{definition}{Определение}
\newtheorem{theorem}{Теорема}
\newtheorem{lemma}{Лемма}
\newtheorem{proposition}{Утверждение}

% Команды для удобства
\newcommand{\F}{\mathbb{F}}
\newcommand{\Z}{\mathbb{Z}}

\begin{document}

Пусть мономы записаны как $x^a = x_1^{a_1} \cdots x_n^{a_n}$.

\textbf{1. Лексикографический порядок}

Мы говорим, что $x^a < x^b$ в порядке lex, если в первой позиции, где $a_i$ и $b_i$ отличаются, выполняется $a_i < b_i$.

(1)
Пусть $x^a < x^b$ и $x^b < x^c$. Тогда в первой позиции, где $a$ и $b$ отличаются, выполнено $a_i < b_i$.
Если $b$ и $c$ отличаются раньше, то там $a_j \le b_j < c_j$, значит $x^a < x^c$. Если же отличие дальше, то в позиции $i$ имеем $a_i < b_i \le c_i$. Поэтому $x^a < x^c$.

(2)
Для любых двух мономов есть первая позиция, где они отличаются
(или они одинаковые). В этой позиции один показатель меньше другого,
поэтому либо $x^a = x^b$, либо $x^a < x^b$, либо $x^b < x^a$.

(3) 
При умножении на один и тот же моном показатели увеличиваются на одинаковые числа, поэтому первая отличающаяся позиция и знак сравнения не меняются. Значит:
$$
x^a < x^b \;\Rightarrow\; x^{a+c} < x^{b+c}.
$$

\textbf{2. Порядок grlex}

Мы говорим, что $x^a < x^b$ в порядке grlex, если сначала
$$
a_1 + \dots + a_n < b_1 + \dots + b_n,
$$
а если суммы равны, то сравниваем как в lex.

(1) 
Если суммы степеней идут в правильном порядке, то транзитивность очевидна.
Если суммы равны, то используется сравнение по lex, который транзитивен.
Во всех случаях порядок остаётся транзитивным.

(2) 
Для любых $a$ и $b$ их суммы степеней либо меньше, либо равны, либо больше. Если суммы равны, то порядок lex сравнивает их. Значит любые два монома сравнимы.

(3)
При умножении суммы степеней увеличиваются на одно и то же число:
$$
(a_1+\dots+a_n) + |c| \quad\text{и}\quad (b_1+\dots+b_n) + |c|.
$$
Поэтому сравнение по сумме не меняется. Если суммы равны, то действует lex, который тоже совместим с умножением. Следовательно:
$$
x^a < x^b \;\Rightarrow\; x^{a+c} < x^{b+c}.
$$

\medskip
\medskip
\medskip
\medskip
\medskip
\medskip

Пусть в кольце $K[x_1,\dots,x_n]$ задан мономиальный порядок.
Пусть есть полином $f$ и порождающие $g_1,\dots,g_m$.
Редукция: если старший моном $g_i$ делит старший член $f$, то делаем шаг
$$
f \;\longrightarrow\; f - \frac{\operatorname{LT}(f)}{\operatorname{LT}(g_i)}\, g_i.
$$
Повторяем, пока ни один $g_i$ не может сократить текущий $f$. Остаток называется нормальной формой $r$.

\textbf{1. Если получился ноль}

Если после последовательности шагов редукции получили $0$, то $f$ принадлежит идеалу $(g_1,\dots,g_m)$.

Пусть последовательность шагов редукции была следующей:
$$
f = f_0 \mapsto f_1 \mapsto f_2 \mapsto \dots \mapsto f_k = 0,
$$
где на каждом шаге $f_{j+1} = f_j - q_j g_{i_j}$ для некоторого $q_j$. Перепишем:
$$
f_0 = f_1 + q_0 g_{i_0}, \quad f_1 = f_2 + q_1 g_{i_1}, \quad \dots \quad f_{k-1} = f_k + q_{k-1} g_{i_{k-1}}.
$$
Складывая эти равенства, получаем
$$
f_0 = q_0 g_{i_0} + q_1 g_{i_1} + \dots + q_{k-1} g_{i_{k-1}} + f_k.
$$
Таким образом, $f = f_0$ представим как линейная комбинацая порождающих $g_i$, значит он принадлежит идеалу.

\textbf{2. Если получился не ноль}

Возможны два случая:

$1.$ $f$ не принадлежит идеалу $(g_1,\dots,g_m)$.

$2.$ $f$ принадлежит идеалу. Тогда можно получить ненулевой остаток, хотя $f \in (g_1,\dots,g_m)$.

Пусть $K[x,y]$ с лексикографическим порядком $x>y$, порождающие
$$
g_1 = x - y, \quad g_2 = y^2 - 1,
$$
$$
f = x^2 - y^2.
$$

Если сначала редуцировать по $g_1$, получаем $0$, но если редуцировать по $g_2$, получаем $x^2 - 1 \neq 0$. Таким образом, $f \in (g_1,g_2)$, но остаток мог оказаться ненулевым.

\textbf{3. Существует ли порядочный подданный, который не сможет пройти ритуал?}

Да, существует. Он описан в примере 2.

\textbf{4. Зависит ли результат ритуала от того, в каком порядке
инквизиторы будут проводить ритуал?}

Да, зависит. Пример 2.

\medskip
\medskip
\medskip
\medskip
\medskip
\medskip

Пусть задан идеал $I = (g_1,\dots,g_m)$ в $K[x_1,\dots,x_n]$ и фиксирован мономиальный порядок.
Множество $\{g_1,\dots,g_m\}$ называется базисом Грёбнера, если
$$
\operatorname{in}(I) = (\operatorname{in}(g_1),\dots,\operatorname{in}(g_m)).
$$

Редукция определяется так же, как раньше: если старший моном $g_i$ делит старший член текущего полинома $f$, то делаем шаг
$$
f \;\longrightarrow\; f - \frac{\operatorname{LT}(f)}{\operatorname{LT}(g_i)}\, g_i.
$$
Повторяем, пока ни один $g_i$ не сокращает $f$. Полученный остаток называется нормальной формой $r$.

\textbf{Любой ли набор полиномов является базисом Грёбнера?} 

Для идеала $I=(g_1,\dots,g_m)$ всегда ли
$$
\operatorname{in}(I) = (\operatorname{in}(g_1),\dots,\operatorname{in}(g_m))\, ?
$$

Нет, не любой набор. В общем случае равенство может не выполняться.

Возьмём
$$
I=(x+y,\;x-y)
$$
и лексикографический порядок $x>y$. Тогда
$$
\operatornamein(x+y)=x,\ \operatorname{in}(x-y)=x,
$$
и
$$
(\operatorname{in}(x+y),\ \operatorname{in}(x-y))=(x).
$$
Но в самом идеале
$$
(x+y)-(x-y)=2y\in I,
$$
и $\operatorname{in}(2y)=y$. Поскольку $y\notin(x)$, имеем
$$
(\operatorname{in}(x+y),\operatorname{in}(x-y))\neq\operatorname{in}(I).
$$
Следовательно $\{x+y,x-y\}$ не является базисом Грёбнера для $I$.

\bigskip
\bigskip
\bigskip

\textbf{Что если
провести ритуал с базисом Грёбнера вместо артефактов? Как
теперь интерпретировать результаты?}


\textbf{1. Если получился ноль}

Если при редукции по базису Грёбнера получился $0$, то $f$ обязательно принадлежит идеалу $I$.

Определение базиса Грёбнера гарантирует, что старший член любого $f\in I$ всегда сокращается каким-то $g_i$. Поэтому редукция будет идти до тех пор, пока не дойдёт до нуля.

\textbf{2. Если получился не ноль}

Если нормальная форма $r\neq 0$, то $f$ не принадлежит идеалу $I$.

Если бы $f$ лежал в идеале, то его старший член обязательно делился бы на один из $\operatorname{in}(g_i)$. Следовательно, редукция продолжалась бы и в итоге дала бы $0$.

\textbf{3. Существует ли порядочный подданный, который не сможет пройти ритуал?}

Нет, не существует. См. пункт 2.

\textbf{4. Зависит ли результат ритуала от того, в каком порядке
инквизиторы будут проводить ритуал?}

Нет. См. пункт 2.

\bigskip
\bigskip
\bigskip

\textbf{Докажите или опровергните утверждение. Любой базис Грёбнера можно привести к редуцированному.}

Пусть $G = \{g_1, \dots, g_s\}$ — произвольный базис Грёбнера идеала $I$.

\textbf{Ведущие коэффициенты равны 1.} 

Для каждого $g_i \in G$ ведущий коэффициент $LC(g_i)$ обратим в поле $K$. Домножив $g_i$ на $LC(g_i)^{-1}$, получаем многочлен с ведущим коэффициентом 1:
$$
g_i \mapsto LC(g_i)^{-1} g_i.
$$
Это сохраняет идеал $I$, так как умножение на обратимый элемент поля не меняет принадлежность многочлена идеалу.

\textbf{Ведущие мономы попарно взаимнопросты.} 

Если существует $i \neq j$, такие что $LM(g_i)$ делится на $LM(g_j)$, то можно заменить
$$
g_i \mapsto g_i - q g_j,
$$
После такой замены $LM(g_i)$ перестанет делиться на $LM(g_j)$. Повторяя процесс для всех пар $i \neq j$, получаем, что ведущие мономы всех элементов базиса попарно не делятся друг на друга.

\textbf{Все ненулевые мономы $g_i$, кроме ведущего не делятся ни на один из ведущих мономов.}

Для каждого ненулевого монома $m$ в $g_i$, кроме ведущего, проверяем, делится ли он на $LM(g_j)$ при $j \neq i$. Если да, то вычитаем соответствующее кратное $g_j$:
$$
g_i \mapsto g_i - c \cdot m' g_j,
$$
Повторяя для всех таких мономов, мы гарантируем, что каждый ненулевой моном $g_i$, кроме ведущего, не делится на ведущие мономы других $g_j$.

Следовательно, любой базис Грёбнера можно привести к редуцированному базису.

\end{document}